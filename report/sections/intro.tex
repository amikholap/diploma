\sectioncentered*{Введение}
\addcontentsline{toc}{section}{Введение}
\label{sec:intro}

В 1886 году Генрих Герц открыл существование электромагнитных волн, а уже через 20 лет с их помощью научились передавать информацию. Так началась эпоха радиосвязи, которая продолжается и в наши дни. Связь без проводов прочно вошла в жизни огромного множества людей, от моряков, получивших возможность общаться с сушей, до большинства из нас, как пользователей мобильных телефонов. Анализ радиоизлучения позволил определять расположение удаленных объектов и даже понять из чего сделаны звезды. Открытие Герца, можно сказать, дало человечеству новый орган восприятия, значительно усиливший потенциал его возможностей.

Традиционно тракт приема и передачи сигнала полностью реализуется в электронных схемах. Для каждой операции --- фильтр, модуляция, демодуляция и пр. --- существует свой физический компонент. Составляя их в последовательности можно получить самые разнообразные функциональные элементы. Такой подход успешно применяется на практике, но он накладывает определенные ограничения на использование и поддержку таких систем. Однажды спроектированная и построенная, она не может изменить свои характеристики в процессе эксплуатации, или изменяет их заранее продуманным способом. Это означает, что старый приемник не сможет обрабатывать новый вид сигналов, а в собранную рацию нельзя добавить новый криптографический алгоритм.

Такое положение вещей было оспорено к концу \Romannum{19} века, когда начались попытки внести гибкость программного обеспечения в мир радиоэлектроники. Функции некоторых элементов приемопередающего тракта воплотили в виде программных компонент. Эта инициатива была поддержана военными США, которые в 1990г. запустили первый публичный проект по внедрению \SDR --- SPEAKeasy \cite{speakeasy_wiki}. Их целью было создание копий существующих раций, которые смогли бы поддерживать новые стандарты связи.

Долгое время оборудование для радиосвязи, особенно нетрадиционные \SDR, стоило достаточно больших денег. Только в последние годы радиолюбители обнаружили, что с помощью платы дешевого DVB-T TV тюнера можно получать радиосигнал в комплексной форме, а значит использовать его как радиоприемник \cite{rtl_sdr_about}. Конечно, качество приема значительно отстает от профессиональных средств, а передатчика в комплекте нет вообще, но это позволило любому желающему проводить свои эксперименты в радиоэфире. В работе используется \SDR-приемник именно такого типа.

В скором времени после появления доступного оборудования для него было написано множество прикладных программ. Появились средства для приема цифрового телевидения, получения и отображения данных метеоспутников и даже интерактивные карты рейсов пассажирских самолетов в окрестности приемника. Фантазия радиолюбителей ограничивается по большей части техническими характеристиками недорогого устройства --- низкой частотой дискретизации и пределами рабочего диапазона частот.

Целью дипломного проекта было выбрано обнаружение сигналов. Это фундаментальная задача во многих областях радиотехники, от проектирования любительских радиоприемников до радиоразведки. Теоретические принципы и методы в этом направлении хорошо исследованы, интерес представляет их практическое применение для широкого класса сигналов с использованием \sdr.

В рамках проекта будет создано программное средство, способное сканировать заданную полосу частот и выделять в ней области присутствия нешумовых сигналов. В качестве радиоприемного устройства используется R820T/R820T2 RTL-SDR. Цифровая обработка сигналов реализована на языке программирования Python. При анализе применены методы математической статистики.

В первой главе говорится об аналогах данного программного средства и его отличиях. Далее дается описание аппаратной части проекта, технические характеристики, возможности и ограничения платформы. В следующей главе объясняется выбор использованных математических методов. Завершает основную часть обзор результатов применения приложения.
