\sectioncentered*{Заключение}
\addcontentsline{toc}{section}{Заключение}

В рамках дипломного проекта было создано программное средство, способное сканировать радиоэфир и выделять в нем информативные сигналы.
Аппаратной базой разработки послужило \sdr. Проанализированы технические характеристики и принципы работы этого типа устройств.
Были реализованы несколько анализаторов сигнала, использующих различные алгоритмы обнаружения. Все они основываются на методах статистики и математического моделирования.

Практическое применение приложения продемонстрировало хорошие результаты. Мощные сигналы быстро определяются, доля ошибок типа "<ложная тревога"> невелика. Возможна тонкая настройка алгоритмов под задачу путем изменения их параметров.
Архитектура взаимодействия компонент системы позволяет легко добавлять новые анализаторы для работы с неизвестными ранее типами сигналов.

Программный продукт не имеет прямых аналогов из-за новизны области приложений для \SDR. Эта область находится в стадии активного развития, ведутся эксперименты по замещению традиционных радиосредств цифровыми.

В результате цель дипломного проекта была достигнута.
Было создано программное обеспечение.
Однако, тема не исчерпала себя. Углубленное ее рассмотрение может привести к более эффективным методам, обеспечивающим более широкое покрытие типов сигналов, встречаемых на практике.

В частности, инструментом заслуживающим внимания являются сверточные нейронные сети. В последние годы они осуществили прорыв в области классификации структур, имеющих закономерности в пространственном расположении, в частности изображений. Применение их к спектрограммам может дать значительно лучшие результаты, чем предопределенный набор эвристик.
