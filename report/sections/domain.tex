\section{Обзор предметной области}
\label{sec:domain}

\subsection{Математическое понимание радиосигнала (1)}

Радиосвязь --- это разновидность беспроводной связи, где носителем сигнала является радиоволна. Передача и прием информации осуществляется посредством излучения и поглощения электромагнитных колебаний.
Передаваемые данные как правило неслучайны и могут быть представлена как функция от полезной информации и времени. В идеальных условиях было бы возможно, имея некоторые априорные знания о характере сигнала, безошибочно восстановить эту функцию и получить полезную информацию без потерь. Для примера можно рассмотреть стандартную запись сигнала с фазовой модуляцией (\autoref{eq:domain:pm_example}).

\begin{equation}
  \label{eq:domain:pm_example}
  s(t) = A_0 cos(\omega_0 t + m s_m(t))
\end{equation}
\begin{explanation}
\item[где] $s_m(t)$ --- полезная информация.
\end{explanation}

Но при распространении электромагнитной волны, она неизбежно испытывает влияние среды и изменяет свои параметры. Кроме того, вместе с целевым сигналом на приемник поступают множество шумовых. В результате полезную информацию можно восстановить только с определенной погрешностью.
Мгновенный уровень сигнала при этом непредсказуемо колеблется, поэтому в каждый момент времени его можно рассматривать как реализацию некоторой случайной величины, а развернутый во времени сигнал как случайный процесс. Такая абстракция очень полезна при статистическом анализе сигнала, так как позволяет использовать известные математические приемы при исследовании. Кроме того, его неизбежная составляющая --- фоновый шум --- обычно имеет нормальное распределение с нулевым математическим ожиданием.

Эта математическая модель обобщается на случай смесей сигналов и разладок (изменений параметров предполагаемого распределения). Также над случайными величинами определены математические операции: сложение, умножение и др. Это значит, что в терминах случайного процесса воспроизводимы как минимум элементарные преобразования сигналов.

\subsection{Типы сигналов (3)}
\subsection{Типы помех (2)}
\subsection{Комплексное представление сигнала (2)}
\subsection{Временной и частотный анализ (3)}
