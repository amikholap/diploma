\thispagestyle{empty}

\begin{singlespace}

{\small
  \begin{center}
    \begin{minipage}{0.8\textwidth}
      \begin{center}
        {\normalsize ОТЗЫВ}\\[1em]
        на дипломный проект студента факультета компьютерных систем и сетей 
        Учреждения образования <<Белорусский государственный университет информатики и радиоэлектроники>>\\
        Михолапа Алеся Александровича \\
        на тему: <<Программное средство для обраружения радиосигналов с помощью \SDR-приемника>>
      \end{center}
    \end{minipage}
  \end{center}

Целью дипломного проекта Михолапа А.А. было создание радиосканера на аппаратной основе \sdr.
В настоящее время наблюдается рост популярности \SDR. Предпринимаются первые шаги по применению этих систем для решения открытых задач радиотехники, таких как обеспечение электромагнитной совместимости радиосредств и возможность модификации их характеристик без изменения аппаратной базы.
Исследования в этом направлении имеют перспективы найти практическое применение в современной радиотехнике.

Выбранная тема разработки программного продукта глубоко пересекается с непрофильными для студентов факультета компьютерных систем и сетей областями цифровой обработки сигналов и радиосвязи. Михолап А.А. самостоятельно изучил специализированную литературу на достаточно глубоком уровне и продемонстрировал способность использовать приобретенные навыки в новой предметной области.

В разработанном приложении применены широко используемые в профессиональной среде технологии программирования, а продуманная архитектура подходит для наращивания функциональности и дальнейшего развития.
Реализованные алгоритмы используют достижения различных прикладных наук: цифровой обработки сигналов, статистики, математического моделирования.

Календарный график в процессе выполнения задания не нарушался. На всех точках контроля объем проделанной работы соответствовал ожидаемому.

Пояснительная записка написана точно, последовательно, со всесторонним пониманием темы. Она содержит необходимые знания о предметной области и полно освещает методы, подходы и инструменты, использованные для решения задачи.
Пояснительная записка и графический материал оформлены в соответствии с требованиями ЕСКД.

Продемонстрированный Михолапом А.А. уровень профессиональной подготовки и способность самостоятельно осваивать новые области знания позволяют считать цель его обучения достигнутой, а его самого достоиным присвоения квалификации математик-системный программист.

  \vfill
  \noindent
  \begin{minipage}{0.54\textwidth}
    \begin{flushleft}
      Руководитель проекта:\\
      кандитат технических наук, доцент,\\
      доцент кафедры информатики\\[0.2em]
      27.05.15
    \end{flushleft}
  \end{minipage}
  \begin{minipage}{0.44\textwidth}
    \begin{flushright}
      \underline{\hspace*{3cm}} М.\,В.~Стержанов
    \end{flushright}
  \end{minipage}
}

\end{singlespace}

\clearpage
