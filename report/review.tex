\thispagestyle{empty}


{\small
  \begin{center}
    \begin{minipage}{0.9\textwidth}
      \begin{center}
        {\normalsize РЕЦЕНЗИЯ}\\[0.2cm]
        на дипломный проект студента факультета компьютерных систем и сетей Учреждения образования <<Белорусский государственный университет информатики и радиоэлектроники>>\\
        Михолапа Алеся Александровича \\
        на тему: <<Программное средство для обнаружения радиосигналов с помощью \SDR-приемника>>
      \end{center}
    \end{minipage}\\
  \end{center}

Дипломный проект студента Михолапа А.А. состоит из шести листов графического материала и~\pageref*{LastPage} страниц пояснительной записки.

Темой проекта выбрано применение программируемого радиоприемника в задаче обнаружения сигналов.
В настоящее время этот тип устройств находится на стадии экспериментального внедрения в различные сферы радиотехники, что говорит о перспективности разработок на его основе.

В пояснительной записке собраны все необходимые сведения по теоретическому обоснованию и практической реализации проекта.
Приведенный материал позволяет получить целостное восприятие проблемы и способов ее решения.
Большое количество иллюстраций гармонично дополняет текст и способствует лучшему пониманию изложенных идей.

Дано описание технических характеристик и принципов работы \sdr.
Произведен обзор предметной области, описаны математические конструкции для представления радиосигналов и различные алгоритмы их обработки.
Подробно изложена архитектура программного средства и реализованные методы обнаружения.

Предложены различные пути решения проблемы, использующие в своей основе математическое моделирование и статистику.
Логично и обоснованно проанализированы их преимущества и недостатки, рассмотрены дальнейшие пути развития проекта.
Разностороннее и подробное изложение свидетельствует об интенсивной работе с литературными источниками из разных областей математики и информатики.

Пояснительная записка и графический материал оформлены аккуратно и в соответствии с требованиями ЕСКД.

Созданное программное средство может использоваться как часть радиотехнической системы.
Предусмотрено расширение функциональности продукта через систему модулей для поддержки новых типов сигналов.

Замечания:
\begin{itemize}
  \item{при сравнении аппаратных средств не указана стабильность частоты;}
  \item{при описании реализации не указан использованный тип окна для оконного преобразования Фурье.}
\end{itemize}

Дипломный проект полно раскрывает обозначенную тему, демонстрирует умение осваивать и работать с современными технологиями и заслуживает оценки десять баллов, а дипломник Михолап А.А. --- присвоения квалификации математик-системный программист.

  \vfill
  \noindent
  \begin{minipage}{0.4\textwidth}
    \begin{flushleft}
      Рецензент:\\
      Преподаватель 1 к. каф. информатики УО\\
      «Минский государственный высший радиотехнический колледж»
    \end{flushleft}
  \end{minipage}
  \begin{minipage}{0.58\textwidth}
    \begin{flushright}
    \underline{\hspace*{3cm}}\hspace*{0.5cm}\underline{\hspace*{2cm}} С.\,Г.~Буянова \\
    Дата\hspace*{6.5cm}
    \end{flushright}
  \end{minipage}
}

\clearpage
