\sectioncentered*{Реферат}
\thispagestyle{empty}

\emph{Ключевые слова}: software-defined radio; цифровая обработка сигналов; обнаружение сигналов; спектральный анализ сигналов; проверка статистических гипотез.

\vspace{4\parsep}

Дипломный проект выполнен на 6 листах формата А1 с пояснительной запиской на~\pageref*{LastPage} страницах, без приложений справочного или информационного характера. 
Пояснительная записка включает \total{section}~глав, \totfig{}~рисунков, \tottab{}~таблиц, \toteq{}~формул и \totref{}~литературный источник.

Целью дипломного проекта является разработка программного средства, использующего аппаратное обеспечение software defined radio для сканирования радиоэфира и обнаружения нешумовых сигналов.

Для достижения цели дипломного проекта было разработано прикладное программное средство на языках \python{} и \purec{}, работающее режимах сканирования радиоэфира и исследования отдельных частот, а также пользовательский интерфейс для интерактивной работы.
Приложение использует различные методы проверки статистических гипотез для обнаружения разных видов радиосигналов.

В разделе технико-экономического обоснования был произведён расчёт затрат на создание ПО, а также прибыли от разработки, получаемой разработчиком.
Проведённые расчёты показали экономическую целесообразность проекта.

Пояснительная записка включает раздел по охране труда, в котором была произведена оценка пожарной безопасности на предприятии, где частично разрабатывался данный дипломный проект.

\clearpage