\sectioncentered*{Аннотация}
\thispagestyle{empty}

\begin{center}
  \begin{minipage}{0.82\textwidth}
    на дипломный проект <<Программное средство для обнаружения радиосигналов с помощью SDR-приемника>> студента УО <<Белорусский государственный университет информатики и радиоэлектроники>> Михолапа~А.\,А.
  \end{minipage}
\end{center}

\emph{Ключевые слова}: \MakeUppercase{software-defined radio; цифровая обработка сигналов; обнаружение сигналов; спектральный анализ сигналов; моделирование случайных процессов}.

\vspace{4\parsep}

Дипломный проект выполнен на 6 листах формата А1 с пояснительной запиской на~\pageref*{LastPage} страницах, без приложений справочного или информационного характера. 
Пояснительная записка включает \total{section}~глав, \totfig{}~рисунков, \tottab{}~таблиц, \toteq{}~формулы, \totref{}~литературный источник.

Целью дипломного проекта является разработка радиосканера на аппаратной базе \sdr, способного производить пассивное сканирование эфира и оповещать о присутствии в нем активных каналов передачи с фильтрацией шумовых сигналов.

Для достижения цели дипломного проекта было разработано программное средство, способное обнаружать нешумовые сигналы в заданной полосе частот.
В нем реализованы различные алгоритмы, учитывающие специфику различных типов сигналов.

Во введении происходит ознакомление с проблемой, решаемой в дипломном проекте и кратко описываются отличия \sdr от традиционных радиосредств.

В первой главе производится обзор предметной области --- основных типов сигналов и помех и математических конструкций, применяемых в задачах обнаружения сигналов.

Во второй главе детальнее рассказывается об особенностях \SDR, его технических характеристиках и основных принципах работы.

В третьей главе приводятся математические алгоритмы, использованные в дипломном проекте.

В четвертой главе описывается реализация программного продукта.
Рассказывается про его архитектуру и детали примененных алгоритмов.

В пятой главе производится технико"=экономическое обоснование разработки.

В шестой главе производится оценка пожарной безопасности предприятия, на котором была пройдена преддипломная практика.

В заключении подводятся итоги и делаются выводы по дипломному проекту, а также описывается дальнейший план развития проекта.

\clearpage
